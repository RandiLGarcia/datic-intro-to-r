\documentclass[english,man]{apa6}

\usepackage{amssymb,amsmath}
\usepackage{ifxetex,ifluatex}
\usepackage{fixltx2e} % provides \textsubscript
\ifnum 0\ifxetex 1\fi\ifluatex 1\fi=0 % if pdftex
  \usepackage[T1]{fontenc}
  \usepackage[utf8]{inputenc}
\else % if luatex or xelatex
  \ifxetex
    \usepackage{mathspec}
    \usepackage{xltxtra,xunicode}
  \else
    \usepackage{fontspec}
  \fi
  \defaultfontfeatures{Mapping=tex-text,Scale=MatchLowercase}
  \newcommand{\euro}{€}
\fi
% use upquote if available, for straight quotes in verbatim environments
\IfFileExists{upquote.sty}{\usepackage{upquote}}{}
% use microtype if available
\IfFileExists{microtype.sty}{\usepackage{microtype}}{}

% Table formatting
\usepackage{longtable, booktabs}
\usepackage{lscape}
% \usepackage[counterclockwise]{rotating}   % Landscape page setup for large tables
\usepackage{multirow}		% Table styling
\usepackage{tabularx}		% Control Column width
\usepackage[flushleft]{threeparttable}	% Allows for three part tables with a specified notes section
\usepackage{threeparttablex}            % Lets threeparttable work with longtable

% Create new environments so endfloat can handle them
% \newenvironment{ltable}
%   {\begin{landscape}\begin{center}\begin{threeparttable}}
%   {\end{threeparttable}\end{center}\end{landscape}}

\newenvironment{lltable}
  {\begin{landscape}\begin{center}\begin{ThreePartTable}}
  {\end{ThreePartTable}\end{center}\end{landscape}}

  \usepackage{ifthen} % Only add declarations when endfloat package is loaded
  \ifthenelse{\equal{\string man}{\string man}}{%
   \DeclareDelayedFloatFlavor{ThreePartTable}{table} % Make endfloat play with longtable
   % \DeclareDelayedFloatFlavor{ltable}{table} % Make endfloat play with lscape
   \DeclareDelayedFloatFlavor{lltable}{table} % Make endfloat play with lscape & longtable
  }{}%



% The following enables adjusting longtable caption width to table width
% Solution found at http://golatex.de/longtable-mit-caption-so-breit-wie-die-tabelle-t15767.html
\makeatletter
\newcommand\LastLTentrywidth{1em}
\newlength\longtablewidth
\setlength{\longtablewidth}{1in}
\newcommand\getlongtablewidth{%
 \begingroup
  \ifcsname LT@\roman{LT@tables}\endcsname
  \global\longtablewidth=0pt
  \renewcommand\LT@entry[2]{\global\advance\longtablewidth by ##2\relax\gdef\LastLTentrywidth{##2}}%
  \@nameuse{LT@\roman{LT@tables}}%
  \fi
\endgroup}


\ifxetex
  \usepackage[setpagesize=false, % page size defined by xetex
              unicode=false, % unicode breaks when used with xetex
              xetex]{hyperref}
\else
  \usepackage[unicode=true]{hyperref}
\fi
\hypersetup{breaklinks=true,
            pdfauthor={},
            pdftitle={The Effect of Biological Sex and Gender Expression on Hireability of Entry-Level Job},
            colorlinks=true,
            citecolor=blue,
            urlcolor=blue,
            linkcolor=black,
            pdfborder={0 0 0}}
\urlstyle{same}  % don't use monospace font for urls

\setlength{\parindent}{0pt}
%\setlength{\parskip}{0pt plus 0pt minus 0pt}

\setlength{\emergencystretch}{3em}  % prevent overfull lines

\ifxetex
  \usepackage{polyglossia}
  \setmainlanguage{}
\else
  \usepackage[english]{babel}
\fi

% Manuscript styling
\captionsetup{font=singlespacing,justification=justified}
\usepackage{csquotes}
\usepackage{upgreek}

 % Line numbering
  \usepackage{lineno}
  \linenumbers


\usepackage{tikz} % Variable definition to generate author note

% fix for \tightlist problem in pandoc 1.14
\providecommand{\tightlist}{%
  \setlength{\itemsep}{0pt}\setlength{\parskip}{0pt}}

% Essential manuscript parts
  \title{The Effect of Biological Sex and Gender Expression on Hireability of
Entry-Level Job}

  \shorttitle{BIOLOGICAL SEX, GENDER EXPRESSION, AND HIREABILITY}


  \author{Connie Zhang\textsuperscript{1}~\& Randi L. Garcia\textsuperscript{1}}

  \def\affdep{{"", ""}}%
  \def\affcity{{"", ""}}%

  \affiliation{
    \vspace{0.5cm}
          \textsuperscript{1} Smith College  }

  \authornote{
    \newcounter{author}
    Complete departmental affiliations for each author (note the
    indentation, if you start a new paragraph). Enter author note here.

                      Correspondence concerning this article should be addressed to Connie Zhang, Postal address. E-mail: \href{mailto:rgarcia@smith.edu}{\nolinkurl{rgarcia@smith.edu}}
                          }


  \abstract{Our study was designed to investigate the effect of biological sex and
gender expressions on hireability of an entry-level job. A sample of 104
participants (most of whom were college women) took an online survey
where they were asked to act as a retail employee in a short vignette
describing a customer. They were then asked to rate the hireability of
the customer. We conducted a 2x2 between-subjects factorial design to
test those effects. We found no significant main effect of biological
sex or gender expressions. There was no interaction effect of biological
sex and gender expressions either. Implications and directions for
future research were discussed.}
  \keywords{keywords \\

    \indent Word count: X
  }




  \usepackage{natbib}

\usepackage{amsthm}
\newtheorem{theorem}{Theorem}
\newtheorem{lemma}{Lemma}
\theoremstyle{definition}
\newtheorem{definition}{Definition}
\newtheorem{corollary}{Corollary}
\newtheorem{proposition}{Proposition}
\theoremstyle{definition}
\newtheorem{example}{Example}
\theoremstyle{remark}
\newtheorem*{remark}{Remark}
\begin{document}

\maketitle

\setcounter{secnumdepth}{0}



Although many people have been calling for gender equality in
employment, gender stereotypes have persisted to disadvantage women in
hiring process, grounded not only on the biological sex but also on the
perceived masculinity and femininity (Harvie, Marshall-Mcaskey, \&
Johnston, 1998). It is of increasing significance to understand how
people's masculine and feminine traits are perceived and evaluated by
others in the hiring process, to add on our knowledge of existing
gender-based discrimination. To this end, the current study uses an
experimental research method to examine how hiring decisions are
impacted by both the biological sex and the gender expressions of
potential employees.

Previous studies have demonstrated that female applicants are viewed as
less hireable than male applicants are (Harvie et al., 1998), and
applicants who show feminine traits are viewed as less hireable than
applicants who show masculine traits (Hareli, Klang, \& Hess, 2008). An
experimental study conducted by Harvie et al. (1998) showed that
participants tended to assign lower-status, lower-salaried jobs to
female applicants compared to male applicants when the participants
themselves acted as job applicants reviewing their peers. However, they
tended to make fairer and more socially desirable decisions when acting
as employers to avoid being labeled as sexist. Hareli et al. (2008)
experimental study indicated that femininity inferred from male
applicants' job history were viewed as an unfavorable characteristic in
the hiring process. In their experiment, male applicants who had had a
gender atypical job were considered less suitable for future gender
typical jobs, although both male and female applicants who had occupied
a job that is stereotypically occupied by the opposite sex were
evaluated as more competent for another gender atypical job. This calls
for explorations into the more complicated gender expressions of humans.
Horvath and Ryan (2003) study on sexual orientation-based discrimination
in the hiring process showed that the direction of discrimination was
more noticeably toward femininity than to non-conforming gender
expressions. In their experiment, participants viewed the resumes of
people indicated as heterosexual and gender conforming or homosexual and
gender non-conforming. The results showed that non-conforming applicants
were evaluated significantly less positively than conforming men but
more positively than conforming women.

Plake et al. (1987) found that breaking gender roles could lead to
positive evaluations. In their experimental study, the researchers found
that, between the two levels of counseling psychologists, directors and
counselors, participants tended to assign applicants with
gender-atypical traits to the leadership roles most possibly because
they were viewed as more flexible and with a wider breadth of skill,
even though all applicants had identical credentials. Contradictory
literature exists regarding this issue because people's non-conformance
of gender and gender roles can be viewed tremendously differently
depending on the extent of viewers' beliefs in traditional gender roles
(Horvath \& Ryan, 2003).

Past literature on hiring bias have demonstrated a general favorability
of male applicants, presented a vague general favorability of masculine
traits (Harvie et al., 1998; Horvath \& Ryan, 2003), and yielded mixed
results of people's attitudes toward biological sex and gender
expression non-conforming applicants. There has been little research
thus far on how biological sex and gender expressions each have impact
on hiring decisions of an entry-level job and how masculinity and
femininity have different extents of effects on each gender. To
investigate this, we sent out a survey with four vignettes each
featuring one customer at the checkout counter of a retail store. The
four customers only differ in biological sex and gender expressions,
manipulated with names and purchases. We predicted that there would be a
main effect of biological sex, such that participants would be more
likely to offer employment opportunity to male customers than female
customers. We also predicted that there would be a main effect of gender
expression, such that participants would be more likely to offer
employment opportunity to customers who showed more masculine traits
than customers who showed more feminine traits. Finally, we predicted
that there would be an interaction effect of biological sex and gender
expression, such that gender expressions would have a larger effect on
male than on female. We thus expected to find that participants would be
more likely to hire masculine female customers than feminine male
customers.

\section{Methods}\label{methods}

\subsection{Design}\label{design}

In order to test the effects of biological sex and perceived gender
presentation on hireability, we used a 2 (biological sex: male, female)
x 2 (gender expression: masculine, feminine) between-subjects factorial
experimental design. The independent variables manipulated in the study
were biological sex and gender expression. Participants were presented
with one of the four short vignettes we created, in which they were
asked to act as the retail employee and decide on whether or not to give
the customer an advertisement for employment opportunities with the
store. The only differences in the vignettes were the biological sex and
gender expression of the customer. The dependent variable was the
likelihood of the customer being hired.

\subsection{Participants}\label{participants}

Participants were recruited through convenience sampling by posting a
link to the online survey both on the Smith College Participant Pool and
on Facebook for anyone to click and share. Of the 109 participants,
6.7\% identified themselves as male, 77.9\% identified themselves as
female, and 7.6\% identified as queer, transgender, or other, 1.9\% of
participants chose not to answer and 5.8\% either left the space blank
or entered an unusable answer. Participant age ranged from 18 to 60,
with an average of 20 and a standard deviation of 5.83. Nine
participants chose not to give their age or entered an unusual answer
(e.g., \enquote{junior}, \enquote{400}, \enquote{0}, \enquote{2019}).
These participants, and those under the age of 18 were not counted. By
using convenience sampling, our sample had a large portion of
participants that identified as females in their late teens. In
addition, 41.3\% of our participants identified as White, 7.7\%
identified as black or African-American, 27.9\% identified as Asian, and
8.7\% of our participants identified as Latino. 5.8\% identified as
Native Americans, while 2.9\% filled in the \enquote{Other} box, mostly
to account for multiracial identities for which we failed to provide an
option. 5.8\% did not answer the race question. After clearing out
unusable responses, 50 participants were assigned to the male customer
condition, 54 were assigned to the female customer condition, 51
participants were assigned to the masculine condition and 53 were
assigned to the feminine condition.

\subsection{Material}\label{material}

To test the hireability of different customers, we created four
vignettes each featuring one particular customer, varying in information
by the different levels of the independent variables (i.e., a masculine
male, a feminine male, a masculine female, a feminine female). We
created a scenario in which the customer casually complains about
something personal associated with the item he or she is intending to
buy. They behave nicely and politely throughout the process of checking
out. To manipulate the gender of the customer, we used the name Michael
for the male and Michelle for the female. For the manipulation of gender
expression, we changed the items the customer bought and the activities
the customer was involved in. Masculinity was indicated by the customer
buying protein shakes and dumbbells and mentioning an injury obtained
working out in the gym preparing for football season. Femininity was
indicated by the customer buying lotion and eyeliner and mentioning his
or her make-up.

\begin{verbatim}
##  raw_alpha
##  0.8665307
\end{verbatim}

Hireability was measured by three questions assessed on a Likert scale
of 1 to 7. The first question was \enquote{How likely are you to give
this person the employment advertisement?} (1 = Not at all likely and 7
= Extremely likely). The second question was \enquote{How much do you
hope this person gets hired?} (1 = Not at all and 7 = Extremely). The
third question was \enquote{How well do you think this person will do if
they are hired?} (1 = Extremely poor and 7 = Extremely well). The three
questions reached high internal consistency (\(\alpha\) = ).
Additionally, participants were asked to rate their customer on seven
traits and the importance of each of the seven traits for a retail
employee, on a Likert scale of 1 to 7 (1 = Not at all and 7 =
Extremely). The seven traits are friendly, talkative, approachable,
efficient, physically strong, considerate and calm under pressure. We
didn't use the answers of the two questions for any analysis.

\subsection{Procedure}\label{procedure}

A questionnaire, via a Qualtrics Survey, was posted on social media
(Facebook) and the Smith College Participant Pool. After the
participants consented and confirmed that they were older than 18, they
got assigned to a random experimental condition and were presented with
a vignette in which the customer is either a masculine male or female or
a feminine male or female. After reading the vignette, the participants
were asked the five above-mentioned questions, three assessing
hireability and two evaluating traits, on a scale ranging from 1 to 7.
Participants were also asked the biological sex and gender expression of
the customer as a manipulation check. They finished the survey by
answering demographic questions on their age, gender, and
race/ethnicity.

\section{Results}\label{results}

In this study, we investigated how biological sex and gender expression
would affect the likelihood of being hired for an entry-level job.
First, we hypothesized that there would be a main effect of biological
sex, such that participants would be more likely to hire a male customer
than a female customer. Second, we hypothesized that there would be a
main effect of gender expression, such that participants would be more
likely to hire a masculine customer than a feminine customer. Finally,
we hypothesized an interaction of biological sex and gender expression,
such that participants would be more likely to hire a masculine woman
than a feminine man.

A two-way ANOVA was used to test if biological sex and gender expression
had an effect on hireability. There was not a statistically significant
main effect of biological sex on hireability, \emph{F} (1, 100) = 0.07,
\emph{p} = 0.79. Participants' scores on hireability of male applicants
(\emph{M} = 4.59, \emph{SD} = 1.32) were higher than participants'
scores on hireability of female applicants (\emph{M} = 4.53, \emph{SD} =
1.12), but not significantly so. There was no statistically significant
main effect of gender expression on hireability, \emph{F} (1, 100) =
0.23, \emph{p} = 0.63. Participants' scores on hireability of feminine
applicants (\emph{M} = 4.62, \emph{SD} = 1.24) were higher than
participants' scores on hireability of masculine applicants (\emph{M} =
4.5, \emph{SD} =1.19), but not significantly so. There was not a
statistically significant interaction of biological sex and gender
expression on hireability, \emph{F} (1, 100) = 1.45, \emph{p} = 0.23.
The four condition means are displayed in Figure 1.

\section{Discussion}\label{discussion}

Our results did not show that biological sex or gender expressions had
any effect on how likely a person got hired. Our results did not show
that there was any interaction of biological sex and gender expressions
on how likely a person got hired either.

In our first hypothesis, we predicted that there would be a main effect
of biological sex, such that male customers would be more likely to get
the employment opportunity than female customers. Our findings did not
support this hypothesis as the result was not found statistically
significant. The results did show that the hireability of male customers
were slightly higher than the hireability of female customers. This is
consistent with findings in the Harvie et al. (1998) study that female
applicants were viewed as less hireable than male applicants when
participants acted as peer employees. The Harvie et al. (1998) study
also showed that when participants were aware of hiring bias against
women, they tended to make fairer decisions to seem unbiased. This might
partly explain why hiring bias against women was not found significant
in our study as it is possible that participants detected the purpose of
our study and gave more socially desirable answers.

In our second hypothesis, we predicted that there would be a main effect
of gender expression, such that customers who showed more masculine
traits would be more likely to get the employment opportunity than
customers who showed more feminine traits. Our findings did not support
this hypothesis. Our results showed that the hireability of feminine
customers were slightly higher than the hireability of masculine
customer, although not significantly so. This is contrary to those found
in Hareli et al. (2008) study which showed that perceived femininity
inferred from male applicants' career history made them less suitable
for future male-typed jobs. The study suggested that this was related to
the belief that jobs that were perceived as suitable for women were also
perceived as less prestigious and tended to pay less than jobs that were
perceived as more suitable for men. In our study, the job (retail
employee) for which the participants were ostensibly recruiting was
supposed to be a gender-neutral job. However, it is still possible that
as an entry-level job, retail employee was viewed as a more feminine
job, thus led participants to rate customers who showed more feminine
traits to be more hireable, though not significantly so.

In our final hypothesis, we predicted that there would be an interaction
effect of biological sex and gender expression, such that participants
would be most likely to hire masculine male customers and least likely
to hire feminine female customers, and more likely to hire masculine
female customers than feminine male customers. This hypothesis was not
supported by our results. Our results showed that feminine male
customers were most likely to be hired and masculine male customers were
least likely to be hired, and masculine female customers were more
likely to be hired than feminine female customers. All the differences
between the scores on hireability were slight and not found significant.
Our findings are contrary to the findings in the study conducted by
Horvath and Ryan (2003) that gender non-conforming applicants were
evaluated less positively than masculine men but more positively than
feminine women, while masculine women and feminine men didn't differ in
scores on hireability. Our results were also contrary to the findings in
the study conducted by Hareli et al. (2008) that male applicants who
showed femininity were viewed as least hireable, since we found feminine
men the most hireable in our study. These two studies both suggested
that this was related to people's beliefs about gender roles. Hareli et
al. (2008) study further suggested that while women have been altering
the boundaries of gender typical jobs by pushing into work domains and
positions traditionally occupied by men, men have not been doing the
same that much, therefore men who have occupied a female sex-typed job
might be perceived as less competent. It is possible, however, that our
findings are different because that was an older study and people's
beliefs in gender roles have changed, over the past few years, and
become generally more favorable to gender and gender role non-conforming
people. It is also possible that people in our sample hold less
conservative beliefs about traditional gender roles than the general
population. Our findings were also supported by the study conducted by
Plake et al. (1987) which found that gender and gender role
non-conforming applicants were viewed more positively than conforming
applicants as those who broke gender roles were viewed as more flexible
and with a wider breadth of skill. This is consistent with our findings
that feminine men and masculine women were rated as more hireable than
feminine women and masculine men, though not significantly so.

There are a number of limitations of our study that must be
acknowledged. First among them is the generalizability of the results.
We used a convenience sample and a large proportion of our participants
were college students who were relatively young, the average of the
participant age being 20. In addition to the age of our participants
being a limitation, 77.9\% of our participants were female, although the
overrepresentation of female in our sample did not lead to a general
favorability of female customers over male customers in results. Our
sample did not accurately represent the population we targeted and thus
caused a decreased external validity. Another limitation is the
manipulation of the gender expressions of fictional customers. We only
used one purchase and one personal fact to indicate each customer as
masculine or feminine and there might not have been enough information
for the participant to form a relatively comprehensive judgment of the
gender expressions of the customer. There is also a limitation about the
measurement of the hireability. We only asked the participants about
their willingness to offer the customer an advertisement for employment
opportunities with the store and that might have been a much more casual
decision than an actual hiring decision. Although our measurement
achieved high reliability, the validity was not ensured.

Future research on the subject of biological sex and gender expressions
in hiring bias will need to use a more representative sample of the
population and include more participants. It would be beneficial to use
resumes to include more information of the potential employees and
ensure the legitimacy of the measurement of hireability. It would also
be important that future research use comparisons of gender-neutral jobs
and sex-typed jobs or entry-level jobs and higher-level jobs, to further
examine the effects of biological sex and gender expressions on hiring
decisions on a larger picture. Furthermore, we also expect to see future
studies look into how beliefs about gender roles could be shaped by
education to mediate hiring discrimination.

Overall our results showed that there were no significant differences
between the hireability of masculine male, feminine male, masculine
female and feminine female. This finding is contrary to some previous
research but could indicate that hiring bias against female, feminine
expressions and gender and gender role non-conforming people has been
decreasing as a whole. We hope that this study, investigating how gender
and gender expressions stimulate hiring bias, will spark future research
on the issue.

\newpage

\section{References}\label{references}

\setlength{\parindent}{-0.5in} \setlength{\leftskip}{0.5in}

\hypertarget{refs}{}
\hypertarget{ref-hareli2008role}{}
Hareli, S., Klang, M., \& Hess, U. (2008). The role of career history in
gender based biases in job selection decisions. \emph{Career Development
International}, \emph{13}(3), 252--269.

\hypertarget{ref-harvie1998gender}{}
Harvie, K., Marshall-Mcaskey, J., \& Johnston, L. (1998). Gender-based
biases in occupational hiring decisions1. \emph{Journal of Applied
Social Psychology}, \emph{28}(18), 1698--1711.

\hypertarget{ref-horvath2003antecedents}{}
Horvath, M., \& Ryan, A. M. (2003). Antecedents and potential moderators
of the relationship between attitudes and hiring discrimination on the
basis of sexual orientation. \emph{Sex Roles}, \emph{48}(3-4), 115--130.

\hypertarget{ref-plake1987access}{}
Plake, B. S., Murphy-Berman, V., Derscheid, L. E., Gerber, R. W.,
Miller, S. K., Speth, C. A., \& Tomes, R. E. (1987). Access decisions by
personnel directors: Subtle forms of sex bias in hiring.
\emph{Psychology of Women Quarterly}, \emph{11}(2), 255--264.






\end{document}
